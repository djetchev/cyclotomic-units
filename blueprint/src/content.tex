% In this file you should put the actual content of the blueprint.
% It will be used both by the web and the print version.
% It should *not* include the \begin{document}
%
% If you want to split the blueprint content into several files then
% the current file can be a simple sequence of \input. Otherwise It
% can start with a \section or \chapter for instance.

\chapter{Introduction}
This is my first experimental mini-project 
to formalize the theory of cyclotomic units in Lean. 


\chapter{Dirichlet $L$-series and class number formulas}\label{chap:dirichlet}

\section{Dirichlet characters}

\begin{definition}[Dirichlet character]\label{defn:dirichlet_character}
    \lean{DirichletCharacter}
    \leanok
    A Dirichlet character modulo $n$ is a group homomorphism $\chi \colon (\mathbb{Z} / n \mathbb{Z})^\times \to \mathbb{C}^\times$. One can easily extend a Dirichlet character 
    to a function $\chi \colon \mathbb{Z} \ra \mathbb{C}^\times$ by the rules 
    $$
    \chi(a) = 0 \text{ if } (a, n) \ne 1, \quad \chi(a + n) = \chi(a).
    $$
    We use this last definition throughout.
\end{definition}


\begin{definition}[Conductor of a Dirichlet character]\label{defn:conductor_dirchar}
    \uses{defn:dirichlet_character}
    \lean{DirichletCharacter.conductor}
    \leanok
    The conductor of a Dirichlet character $\chi$ is the smallest positive integer $f$ such that $\chi(a + f) = \chi(a)$.
\end{definition}

\begin{definition}[Gauss sum]\label{defn:gauss_sum}
    \uses{defn:dirichlet_character}
    \lean{gaussSum}
    \leanok
    Let $\chi$ be a Dirichlet character modulo $f$. The Gauss sum of $\chi$ is defined as 
    $$
    \tau(\chi) = \sum_{a = 1}^f \chi(a) \zeta_f^a,
    $$
    where $\zeta_f$ is a primitive $f$th root of unity.
\end{definition}

\section{Dirichlet $L$-series}

Let $\chi$ be a Dirichlet character of conductor $f$ and let $L(s, \chi)$ 
be the Dirichlet $L$-series attached to $\chi$.

\begin{theorem}\label{thm:L_series_sum_log_units}
    \uses{defn:dirichlet_character, defn:conductor_dirchar, defn:gauss_sum}
    We have 
    $$
    L(1, \chi) = 
    \begin{cases}
    -\frac{\tau(\chi)}{f} \sum_{a = 1}^{f} \overline{\chi}(a) \log | (1 - \zeta_f)^a | & \text{ if } \chi(-1) = 1, \ \chi \ne 1, \\
    \pi i \frac{\tau(\chi)}{f} \cdot \frac{1}{f} \sum_{a = 1}^f \overline{\chi}(a)a & \text{ if } \chi(-1) = -1. 
    \end{cases}
    $$
\end{theorem}

Before we prove the theorem, we need some auxiliary lemmas: 

\begin{lemma}\label{lem:chi_gauss_sum}
    \uses{defn:dirichlet_character, defn:conductor_dirchar, defn:gauss_sum}
    For every integer $b$, we have 
    $$
    \chi(b) \tau(\overline{\chi}) = \sum_{a = 1}^f \overline{\chi}(a) e^{\frac{2\pi i a b}{f}}.
    $$
\end{lemma}

\begin{proof}
    \proves{lema:chi_gauss_sum}
\end{proof}

\begin{proof}[Proof of Theorem~\ref{thm:L_series_sum_log_units}]
    \proves{thm:L_series_sum_log_units}
    \uses{lem:chi_gauss_sum}
    Using Lemma~\ref{lem:chi_gauss_sum} and the definition of the Dirichlet $L$-series, we write 
    \begin{eqnarray*}
        L(1, \chi) &=& \sum_{n = 1}^{\infty} \frac{\chi(n)}{n} = \sum_{n = 1}^{\infty} \frac{1}{n} \frac{1}{\tau(\overline{\chi})} \sum_{a = 1}^f \overline{\chi}(a) e^{2\pi i a n / f} = \\
        &=& \frac{1}{\tau(\overline{\chi})} \sum_{a = 1}^f \overline{\chi}(a) \sum_{n = 1}^{\infty} \frac{e^{2\pi i a n / f}}{n} = \\
        &=& - \frac{1}{\tau(\overline{\chi})} \sum_{a = 1}^f \overline{\chi}(a) \log(1 - \zeta_f^a).
    \end{eqnarray*}
\end{proof}


\chapter{Regulators of Number Fields}
This chapter presents basic definitions and results on regulators of number fields.

Let $L$ be a number field with $r_1$ real and $r_2$ complex embeddings and let $r = r_1 + r_2 - 1$. 

\section{Definition of the regulator}\label{sec:regulator}

Let $\epsilon_1, \dots, \epsilon_r$ be a set of independent units of $L$. 

\begin{definition}\label{defn:mixed_embeddings}
    \lean{NumberField.mixedEmbedding}
    \leanok
    Let $\sigma_1, \dots, \sigma_{r_1}$ 
    be the set of real embeddings and let $(\sigma_{r_1 + 1}, \overline{\sigma}_{r_1 + 1}), \dots, (\sigma_{r}, \overline{\sigma}_{r})$ be the 
    pairs of conjugate complex embeddings.
\end{definition}

\begin{definition}\label{defn:regulator}
    \lean{NumberField.Units.regulator}
    \leanok
    \uses{defn:mixed_embeddings}
    We define the regulator of the units $\epsilon_1, \dots, \epsilon_r$ of $L$ as 
    $$
    R_L(\epsilon_1, \dots, \epsilon_r) := \sqrt{\det(\delta_{i} \log | \sigma_i(\epsilon_j) | )_{i, j}}, 
    $$
    where 
    $$
    \delta_i = 
    \begin{cases}
    1 & \text{ if } \sigma_i \text{ is real}, \\
    2 & \text{ if } \sigma_i \text{ is complex}.
    \end{cases}
    $$
\end{definition}

\section{The index of units subgroups and regulators}

\begin{lemma}\label{lem:index_eq_reg_ratio}
    \uses{defn:regulator}
    Let $\epsilon_1, \dots, \epsilon_r$ be independent units of a number field $K$ that generate a subgroup 
    $A$ of the units of $K$ modulo roots of unity and let $\eta_1, \dots, \eta_r$ generate a subgroup 
    $B$. If $A \subseteq B$ is of finite index then 
    $$
        [B : A] = \frac{R_K(\epsilon_1, \dots, \epsilon_r)}{R_K(\eta_1, \dots, \eta_r)}.
    $$
\end{lemma}

We derive the lemma from the definition of the regulator and the theory of elementary divisors for 
$\mathbb{Z}$-modules.

\begin{lemma}[Smith normal form]\label{lem:elem_divisors}
    \lean{Basis.SmithNormalForm}
    \leanok
    Let $M \in \GL_{n \times m}(\mathbb{Z})$ be an integer matrix. 
    There exists matrices $U \in \GL_n(\mathbb{Z})$ and 
    $V \in \GL_m(\mathbb{Z})$ such that 
    $$
    U M V = 
    \begin{pmatrix}
        D & 0 \\
        0 & 0 
    \end{pmatrix}
    $$ 
    where $D = \textrm{diag}(d_1, \dots, d_r)$ is an $r \times r$ 
    diagonal integer matrix with $d_1 \mid d_2 \mid \dots \mid d_r$.  
\end{lemma}

\begin{proof}
    \proves{lem:elem_divisors}
    \lean{Basis.SmithNormalForm}
    \leanok
\end{proof}

\begin{proof}
    \uses{lem:elem_divisors}
    \proves{lem:index_eq_reg_ratio}
    Since $A \subseteq B$, one can write 
    $$
    \eta_i = \epsilon_1^{a_{i1}} \cdot \dots \cdot \epsilon_r^{a_{i r}} \cdot \mu^{(i)}, \quad i = 1, \dots, r, 
    $$
    where $a_{ij} \in \mathbb{Z}$ and $\mu^{(i)}$ is a root of unity. By taking absolute values and logarithms, 
    $$
    \delta_j \log | \sigma_j(\eta_i) | = \sum_{k = 1}^r a_{ik} \delta_j \log | \sigma_j(\epsilon_k) |, \quad i = 1, \dots, r. 
    $$
    Therefore, using Defintiion~\ref{defn:regulator}, we have
    \begin{equation}\label{eq:reg_ratio}
    \frac{R_K(\epsilon_1, \dots, \epsilon_r)}{R_K(\eta_1, \dots, \eta_r)} = \det ((a_{ik})_{i, k = 1}^r)
    \end{equation}
    Applying Lemma~\ref{lem:elem_divisors}, we can find matrices $U, V \in \GL_2(\mathbb{Z})$ such that 
    $$
    U (a_{ik})_{i, k = 1}^r V = \text{diag}(d_1, \dots, d_r). 
    $$
    Therefore, changing the basis of $A$ and $B$ with $M$ and $N$, respectively, we obtain bases 
    $x_1, \dots, x_r$ of $A$ and $y_1, \dots, y_r$ of $B$ such that $x_i = d_i y_i$, i.e., 
    $$
    B / A = \bigoplus_{i = 1}^r \mathbb{Z} / d_i \mathbb{Z}. 
    $$
    Hence, $[B : A] = |\prod_{i = 1}^r d_i | = \det ((a_{ik})_{i, k = 1}^r)$ which, combined with \eqref{eq:reg_ratio}, 
    proves the lemma.  
\end{proof}


\chapter{The Index of Cyclotomic Units}

\section{Maximal real subfield}
\begin{definition}\label{defn:K_plus}
    Let $K = \mathbb{Q}(\zeta_n)$ where $\zeta_n$ is a primitive $n$-th root of unit. 
    Define the subfield $K^+ := \mathbb{Q}(\zeta_n + \zeta_n^{-1})$ of $K$.
\end{definition}

\begin{lemma}\label{lem:K_plus_max_real_sub}
    \uses{defn:K_plus}
    The subfield $K^+$ is the maximal real subfield of $K$.
\end{lemma}

\begin{proof}
\end{proof}

\begin{lemma}\label{lem:gal_K_plus}
    \uses{defn:K_plus}
    The Galois group of $\Gal(K^+ / \mathbb{Q})$ is generated by the 
    automorphisms $\sigma_a \colon \zeta \mapsto \zeta^a$ for $1 \leq a \leq p^m / 2$, $(a, p) = 1$. 
\end{lemma}

\begin{proof}
    \proves{lem:gal_K_plus}
\end{proof}

\section{Cyclotomic units}
\begin{definition}\label{defn:cyclo_units}
    Define the cyclotomic units of a cyclotomic field $K = \mathbb{Q}(\zeta_n)$ with ring of integers 
    $O_K = \mathbb{Z}[\zeta_n]$ as the group 
    $$
    C_K := O_K^\times \cap \langle \zeta_n^a, 1 - \zeta_n^b \colon a, b \in \mathbb{Z} / n \mathbb{Z}\rangle. 
    $$
\end{definition}

\section{Index theorem}
\begin{theorem}\label{thm:index_of_units}
    \notready
    \uses{defn:K_plus, defn:cyclo_units}
    Let $n := p^m \in \mathbb{N}$ for a prime $p$ and a positive integer $m$, and let $K = \mathbb{Q}(\zeta_n)$ where $\zeta_n$ is a primitive $n$-th root of unit. 
    Let $O_K$ be the ring of integers of $K$. 
    Let $C_K$ be the group of cyclotomic units in $K$ according to Definition~\ref{defn:cyclo_units}.
    Then
    \[
        [O_{K^+}^\times : C_{K^+}] = h_{K^+},
    \]
    where $h_{K^+}$ is the class number of the maximal real subfield $K^+$ of $K$.
\end{theorem}

\begin{proof}
    \proves{thm:K_plus}
    \uses{lem:reg_Ra}
    \begin{itemize}
        \item Show that the regulator of the units $\xi_a$ is non-zero, where 
        $$
        \xi_a := \zeta_{p^m}^{(1-a)/2} \frac{1 - \zeta_{p^m}^a}{1 - \zeta_{p^m}}, \qquad 1 \leq a < p^m / 2, \ (a, p) = 1.
        $$
        \item Compute the regulator $R(\{\xi_a\})$. 
    \end{itemize}
\end{proof}

\subsection{Computing the regulator for $K^+$}

\begin{lemma}\label{lem:reg_Ra}
    \uses{defn:regulator, defn:K_plus}
    We have
    $$
    R(\{\xi_a\}) = h^+ R^+, 
    $$
    where $R^+$ is the regulator of $K^+$.
\end{lemma}

Let $G = \Gal(K^+ / \mathbb{Q})$. We will use the following auxiliary lemma:

\begin{lemma}\label{lem:reg_Ra_charsum}
    \uses{defn:regulator, defn:K_plus}
    We have 
    $$
    | R(\{\xi_a\}) | = \frac{1}{2} \left | \prod_{\substack{\chi \in \widehat{G} \\ \chi \ne 1}} \sum_{a = 1}^{p^m} \chi(a) \log |1 - \zeta^a| \right |. 
    $$
\end{lemma}

\begin{proof}
    \proves{lem:reg_Ra_charsum}
\end{proof}

In addition, we need the following little lemma on roots of unity: 

\begin{lemma}\label{lem:prod_cycl_units}
    For any $1 \leq k < n$, we have 
    $$
    \prod_{\substack{1 \leq a < p^n \\ a \equiv b \bmod p^k}} (1 - \zeta_{p^n}^a) = 1 - \zeta_{p^k}^b.   
    $$
\end{lemma}

\begin{proof}
    \proves{lem:prod_cycl_units}
    We rewrite 
    $$
    \prod_{\substack{1 \leq a < p^n \\ a \equiv b \bmod p^k}} (1 - \zeta_{p^n}^a) = \prod_{s = 0}^{p^{n-k}-1}(1 - \zeta_{p^n}^b \zeta_{p^{n-k}}^s) = 1 - \zeta_{p^k}^b,
    $$
    where the last equality follows from  
    $$
    \prod_{i = 0}^{m-1} (1 - \zeta_m^i x)= x^m \prod_{i = 0}^{m-1} (x^{-1} - \zeta_m^i) = x^m (x^{-m} - 1) = 1 - x^m
    $$
    applied to $m = p^{n-k}$ and $x = \zeta_{p^n}^b$. 
\end{proof}

\begin{proof}[Proof of Lemma~\ref{lem:reg_Ra}]
    \proves{lem:reg_Ra}
    \uses{lem:prod_cycl_units, lem:gal_K_plus, lem:reg_Ra_charsum, thm:L_series_sum_log_units}
    First, for a Dirichlet character $\chi$ modulo $p^n$ of conductor 
    $f_{\chi} = p^k$ for some $1 \leq k < n$, Lemma~\ref{lem:prod_cycl_units} yields
    $$
    \prod_{\substack{1 \leq a < p^n \\ a \equiv b \bmod p^k}} (1 - \zeta_{p^n}^a) = 1 - \zeta_{p^n}^b.   
    $$
    Applying Lemma~\ref{lem:reg_Ra_charsum} and Theorem~\ref{thm:L_series_sum_log_units}, we obtain 
    \begin{eqnarray*}
        % R(\{\xi_a\}) &=& \frac{1}{2} \left | \prod_{\substack{\chi \in \widehat{G} \\ \chi \ne 1}} \sum_{a = 1}^{p^n} \chi(a) \log |1 - \zeta^a| \right | = \\
        % &=& \frac{1}{2} \prod_{\substack{\chi \in \widehat{G} \\ \chi \ne 1}} \sum_{a = 1}^{p^m} \chi(a) \log |1 - \zeta^a| = \\
        % &=&
        R(\{\xi_a\}) &=& \frac{1}{2} \left | \prod_{\chi \in \widehat{G}, \chi \ne 1} \sum_{a = 1}^{p^n} \chi(a) \log |1 - \zeta^a| \right | = \\
        &=& \frac{1}{2} \prod_{\chi \in \widehat{G}, \chi \ne 1} \sum_{a = 1}^{p^m} \chi(a) \log |1 - \zeta^a| = \\
    \end{eqnarray*}
\end{proof}

